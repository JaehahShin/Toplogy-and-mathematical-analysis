\documentclass{article}
\usepackage{amsmath}
\usepackage{amssymb}
\usepackage[colorlinks=true,urlcolor=blue]{hyperref}
\usepackage{listings}
\usepackage{graphicx} % Required for inserting images
\usepackage[section]{placeins}
\usepackage[utf8]{inputenc}
\usepackage[T1]{fontenc} 
\usepackage{xcolor}
\usepackage{pgfplots}
\usepackage{quantikz}
\usepackage{tikz}
\usepackage{tikz-3dplot}

\title{\underline{\textbf{Learning Topology}}}
\author{Jaehah Shin}

\begin{document}
\maketitle
\tableofcontents

\section{Introduction}
I am Jaehah Shin. \\
I am a third year Engineering Science student at the University of Toronto. My major is Robotics engineering. In this document, I summarize what I learn from the textbook \textit{Topology}. I refer to the textbook \textbf{\textit{Topology } by James Munkres.} As I refer to this textbook and summarize it, the flow of this document is similar to the flow of that textbook. \\
The purpose of this document is to learn the concept of topology during summer, and remember what I learned during the summer break. \\
Important aspect to note is, this is what and how I understood through this book. 

\section{Set theory and logic}
\subsection{Fundamental Concepts}
The first thing to learn is basic notation. Usually, we use the following notation.
Capital letters are used to denote sets, and lowercase letters are used to denote elements of sets.
If an object $x$ is an element of a set $A$, we write $x \in A$. If $x$ is not an element of $A$, we write $x \notin A$.
Equality symbol $=$ is used to mean \textit{logical identity}. For example, if $A = B$, then $A$ and $B$ are the same set. \\
If they are different sets, we write $A \neq B$. \\
When A is a \textit{subset} of B, we write $A \subseteq B$. If A = B, then both $A \subseteq B$ and $B \subseteq A$ are true.\\
If $A \subseteq B$ and A is not equal to B, then we say A is proper subset of B, and write $A \subsetneq B$.
$\subset$ is called as inclusion symbol, and $\subseteq$ is called as proper inclusion symbol. Sets can be defined by listing their elements between braces. For example, $A = \{1, 2, 3\}$ is a set A that contains 1, 2, and 3. \\
However, sometimes it is not possible to list all elements of a set. In this case, we can define a set by specifying a property that its elements must satisfy. For example, $A = \{x \in \mathbb{R} \mid x > 0\}$ is a set A that contains all positive real numbers. \\
For reference, $\mid$ is read as "such that". \\

\subsection{Union of sets and def'n of \textit{or}}



\end{document}

